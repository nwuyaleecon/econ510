%Jennifer Pan, August 2011

\documentclass[10pt,letter]{article}
	% basic article document class
	% use percent signs to make comments to yourself -- they will not show up.

\usepackage{amsmath}
\usepackage{amssymb}
\usepackage{tikz}
\usepackage{enumitem}
	% packages that allow mathematical formatting

\usepackage{graphicx}
	% package that allows you to include graphics

\usepackage{setspace}
	% package that allows you to change spacing

\onehalfspacing
	% text become 1.5 spaced

\usepackage{fullpage}
	% package that specifies normal margins

\renewcommand{\vector}[1]{\boldsymbol{#1}}
\newcommand{\problem}[1]{\section*{Problem #1}}
\newcommand{\problempart}[1]{\paragraph{#1}}

\begin{document}
	% line of code telling latex that your document is beginning


\title{Problem Set 4}

\author{Nicholas Wu}

\date{Fall 2020}
	% Note: when you omit this command, the current dateis automatically included

\maketitle
	% tells latex to follow your header (e.g., title, author) commands.
\textbf{Note:} I use bold symbols to denote vectors and nonbolded symbols to denote scalars. I primarily use vector notation to shorthand some of the sums, since many of the sums are dot products.

\problem{1}

\problempart{(1)}
The F-firms are the normal type we have seen:
\[ R = \alpha k_F^{\alpha - 1}n_F^{1-\alpha} \]
\[ w = (1- \alpha) k_F^{\alpha}n_F^{-\alpha} \]
Since the measure of the F firms is 1
\[ \frac{K_F}{N_F} = \left(\frac{\alpha}{R}\right)^{1/(1-\alpha )} \]
\[ \frac{K_F}{Y} = \frac{k_f}{k_F^{\alpha}n_F^{1-\alpha}} = \frac{1}{k_F^{\alpha-1}n_F^{1-\alpha}} = \frac{\alpha}{R} \]
\problempart{(2)}
We already showed the equilibrium wage for the $F$-firm above:
\[ w = (1- \alpha) k_F^{\alpha}n_F^{-\alpha} = (1-\alpha) \left(\frac{\alpha}{R} \right)^{\alpha/(1-\alpha)}\]
Since this wage is competitive, this also must satisfy the wage FOC for maximization of the E-firm:
\[ w = (1-\alpha)(1-\psi)\chi^{1-\alpha}k_{E}^\alpha n_E^{-\alpha}  \]
\[ = (1-\alpha)(1-\psi)\chi^{1-\alpha}(\kappa s_{E})^\alpha (\kappa n_E)^{-\alpha}  \]
\[ = (1-\alpha)(1-\psi)\chi^{1-\alpha}S_E^\alpha N_E^{-\alpha}  \]
Using the wage expression, we get
\[ N_E^\alpha = \frac{(1-\alpha)(1-\psi)\chi^{1-\alpha}S_E^\alpha}{(1-\alpha) \left(\frac{\alpha}{R} \right)^{\alpha/(1-\alpha)}} = \frac{(1-\psi)\chi^{1-\alpha}S_E^\alpha}{\left(\alpha/R \right)^{\alpha/(1-\alpha)}} \]
So
\[ N_E = \frac{(1-\psi)^{1/\alpha}\chi^{(1-\alpha)/\alpha}S_E}{\left(\alpha/R \right)^{1/(1-\alpha)}} \]
\[ N_F = 1 - \frac{(1-\psi)^{1/\alpha}\chi^{(1-\alpha)/\alpha}S_E}{\left(\alpha/R \right)^{1/(1-\alpha)}} \]
\[ m = \psi S_E^\alpha \chi^{1-\alpha} N_E^{1-\alpha}\kappa^{-1}\]
\[ = \psi S_E^\alpha \chi^{1-\alpha} \frac{R(1-\psi)^{(1-\alpha)/\alpha}\chi^{(1-\alpha)^2/\alpha}S_E^{1-\alpha}}{\alpha \kappa} \]
\[ = \frac{R\psi (1-\psi)^{(1-\alpha)/\alpha}\chi^{(1-\alpha)/\alpha}S_E}{\alpha \kappa} \]
\problempart{(3)}
The profit is
\[ (1-\psi)\chi^{1-\alpha}s_{E}^\alpha n_E^{1-\alpha} - wn_E \]
\[ = (1-\psi)\chi^{1-\alpha}S_{E}^\alpha N_E^{1-\alpha}\kappa^{-1} - w\kappa^{-1}N_E \]
\[ = (1-\psi)\chi^{1-\alpha}S_{E}^\alpha \frac{(1-\psi)^{(1-\alpha)/\alpha}\chi^{(1-\alpha)^2/\alpha}S_E^{1-\alpha}}{\alpha/R }\kappa^{-1} - (1-\alpha) \left(\frac{\alpha}{R} \right)^{\alpha/(1-\alpha)}\kappa^{-1}\frac{(1-\psi)^{1/\alpha}\chi^{(1-\alpha)/\alpha}S_E}{\left(\alpha/R \right)^{1/(1-\alpha)}} \]
\[ = RS_{E} \frac{(1-\psi)^{1/\alpha}\chi^{(1-\alpha)/\alpha}}{\alpha\kappa } - (1-\alpha) R \frac{(1-\psi)^{1/\alpha}\chi^{(1-\alpha)/\alpha}S_E}{\alpha\kappa  } \]
\[ = RS_{E} \frac{(1-\psi)^{1/\alpha}\chi^{(1-\alpha)/\alpha}}{\kappa }  \]
\[ = Rs_{E}(1-\psi)^{1/\alpha}\chi^{(1-\alpha)/\alpha}  \]
So the return rate on savings $s_E$ is
\[ R(1-\psi)^{1/\alpha}\chi^{(1-\alpha)/\alpha} \]
\problempart{(4)}
The utility is equivalent to Cobb-Douglas between the two period goods, so each type of consumer saves $\beta/(1+\beta)$ fraction of their income.
\[ s_W = \frac{\beta}{1+\beta} w \]
\[ s_E = \frac{\beta}{1+\beta} m \]
\problempart{(5)}
We need the coefficient of $R$ for the return rate on saavings from part (3) to be greater than 1. That is,
\[(1-\psi)^{1/\alpha}\chi^{(1-\alpha)/\alpha} > 1\]
taking the $\alpha$ power of both sides
\[ (1-\psi) \chi^{1-\alpha} > 1\]
\[ \chi^{1-\alpha} > \frac{1}{1-\psi} \]
The LHS is the productivity factor of the E-firm, so we need this productivity to offset the cost of paying the manager using the firm profit. If this condition fails, no one will invest their savings into E-firms, and hence no E-firms will operate after the next period because they will have no capital.
\problempart{(6)}

\begin{enumerate}[label=(\roman*)]
\item We first find $N_{Et}, N_{Ft}$:
\[ N_{Et} = \frac{(1-\psi)^{1/\alpha}\chi^{(1-\alpha)/\alpha}S_E}{\left(\alpha/R \right)^{1/(1-\alpha)}} = \frac{(1-\psi)^{1/\alpha}\chi^{(1-\alpha)/\alpha}K_{Et}}{\left(\alpha/R \right)^{1/(1-\alpha)}} \]
\[ N_{Ft} = 1 - \frac{(1-\psi)^{1/\alpha}\chi^{(1-\alpha)/\alpha}K_{Et}}{\left(\alpha/R \right)^{1/(1-\alpha)}} \]
Then
\[ Y_{Et} = \chi^{1-\alpha} K_{Et}^\alpha N_{Et}^{1-\alpha} \]
\[ =  (R/\alpha)(1-\psi)^{(1-\alpha)/\alpha}\chi^{(1-\alpha)/\alpha}K_{Et}\]
And from part (1):
\[ \frac{K_{Ft}}{N_{Ft}} = \left(\frac{\alpha}{R}\right)^{1/(1-\alpha )} \]
\[ K_{Ft} = \left(\frac{\alpha}{R}\right)^{1/(1-\alpha )} N_{Ft} \]
\[ Y_{Ft} = \left(\frac{\alpha}{R}\right)^{\alpha/(1-\alpha)} N_{Ft} \]
\[ = \left(\frac{\alpha}{R}\right)^{\alpha/(1-\alpha)} \left(1 - \frac{(1-\psi)^{1/\alpha}\chi^{(1-\alpha)/\alpha}K_{Et}}{\left(\alpha/R \right)^{1/(1-\alpha)}} \right) \]
So total output
\[ Y_t = Y_{Et} + Y_{Ft} = (R/\alpha)(1-\psi)^{(1-\alpha)/\alpha}\chi^{(1-\alpha)/\alpha}K_{Et} + \left(\frac{\alpha}{R}\right)^{\alpha/(1-\alpha)} \left(1 - \frac{(1-\psi)^{1/\alpha}\chi^{(1-\alpha)/\alpha}K_{Et}}{\left(\alpha/R \right)^{1/(1-\alpha)}} \right) \]
\[ = (R/\alpha)(1-\psi)^{(1-\alpha)/\alpha}\chi^{(1-\alpha)/\alpha}K_{Et} + \left(\frac{\alpha}{R}\right)^{\alpha/(1-\alpha)}  - (R/\alpha)(1-\psi)^{1/\alpha}\chi^{(1-\alpha)/\alpha}K_{Et}  \]
\[ = (R/\alpha)\chi^{(1-\alpha)/\alpha}K_{Et} \left((1-\psi)^{(1-\alpha)/\alpha} - (1-\psi)^{1/\alpha} \right) + \left(\frac{\alpha}{R}\right)^{\alpha/(1-\alpha)}   \]
\[ = (R/\alpha)\psi(1-\psi)^{(1-\alpha)/\alpha}\chi^{(1-\alpha)/\alpha}K_{Et}  + \left(\frac{\alpha}{R}\right)^{\alpha/(1-\alpha)}   \]
\item The law of motion for $K_{Et}$:
\[ K_{E,t+1} = \kappa s_E = \kappa \frac{\beta}{1+\beta} m \]
\[ =  \frac{\beta}{1+\beta} \psi K_{Et}^\alpha \chi^{1-\alpha} N_{Et}^{1-\alpha} \]
\[ = \frac{\beta}{1+\beta} \psi (R/\alpha)(1-\psi)^{(1-\alpha)/\alpha}\chi^{(1-\alpha)/\alpha}K_{Et} \]
\item Finally, we can slog through the algebra to compute out $\rho_t$:
\[ \rho_t = \frac{K_{Et}}{K_{Et} + K_{Ft}}\rho_E + \frac{K_{Ft}}{K_{Et} + K_{Ft}}\rho_F \]
\[ = \frac{K_{Et}}{K_{Et} + K_{Ft}}(\alpha\chi^{1-\alpha}K_{Et}^{\alpha-1}N_{Et}^{1-\alpha}) + \frac{K_{Ft}}{K_{Et} + K_{Ft}}(\alpha K_{Ft}^{\alpha-1}N_{Ft}^{1-\alpha}) \]
\[ = \frac{\alpha Y_{Et}}{K_{Et} + K_{Ft}} + \frac{\alpha Y_{Ft}}{K_{Et} + K_{Ft}} \]
\[ = \frac{\alpha Y_{t}}{K_{Et} + K_{Ft}}  \]
\[ = \frac{R\psi(1-\psi)^{(1-\alpha)/\alpha}\chi^{(1-\alpha)/\alpha}K_{Et}  + \alpha \left(\frac{\alpha}{R}\right)^{\alpha/(1-\alpha)} }{K_{Et} + \left(\frac{\alpha}{R}\right)^{1/(1-\alpha )} N_{Ft}}  \]
\[ = \frac{R\psi(1-\psi)^{(1-\alpha)/\alpha}\chi^{(1-\alpha)/\alpha}K_{Et}  + \alpha \left(\frac{\alpha}{R}\right)^{\alpha/(1-\alpha)} }{K_{Et} + \left(\frac{\alpha}{R}\right)^{1/(1-\alpha )} \left(1 - \frac{(1-\psi)^{1/\alpha}\chi^{(1-\alpha)/\alpha}K_{Et}}{\left(\alpha/R \right)^{1/(1-\alpha)}} \right)}  \]
\[ = \frac{R\psi(1-\psi)^{(1-\alpha)/\alpha}\chi^{(1-\alpha)/\alpha}K_{Et}  + \alpha \left(\frac{\alpha}{R}\right)^{\alpha/(1-\alpha)} }{K_{Et} + \left(\frac{\alpha}{R}\right)^{1/(1-\alpha )} - (1-\psi)^{1/\alpha}\chi^{(1-\alpha)/\alpha}K_{Et}}  \]
\[ = \frac{R\psi(1-\psi)^{(1-\alpha)/\alpha}\chi^{(1-\alpha)/\alpha}K_{Et}  + \alpha \left(\frac{\alpha}{R}\right)^{\alpha/(1-\alpha)} }{K_{Et} + \left(\frac{\alpha}{R}\right)^{1/(1-\alpha )} - (1-\psi)^{1/\alpha}\chi^{(1-\alpha)/\alpha}K_{Et}}  \]
\end{enumerate}
\problempart{(7)}
The wages for normal workers is
\[ w = (1-\alpha) \left(\frac{\alpha}{R} \right)^{\alpha/(1-\alpha)} \]
from the firm FOC. This is thus constant over time, since the FOCs for the F-firms fully determines this wage.
\problempart{(8)}
From part (6), we see the law of motion for $K_{Et}$ can be rewritten:
\[ \frac{K_{E,t+1}}{K_{E,t}} = \frac{\beta}{1+\beta} \psi (R/\alpha)(1-\psi)^{(1-\alpha)/\alpha}\chi^{(1-\alpha)/\alpha} \]
Thus $K_{E}$ has constant growth rate. If the RHS is $< 1$, then the stock of E-firm capital must decrease to 0, and in the long run no E-firms will operate. If the RHS is $>1$, then the stock of E-firm capital is growing, and since
\[ N_{Et} =  \frac{(1-\psi)^{1/\alpha}\chi^{(1-\alpha)/\alpha}}{\left(\alpha/R \right)^{1/(1-\alpha)}}K_{Et} \]
Since $N_{Et}$ is linear in $K_{Et}$ which grows at a constant rate, eventually $N_{Et} = 1$ due to the labor constraint. This implies $N_{Ft} = 0$ eventually, and since no workers work for F-firms, all the F-firms stop operating. Therefore, depending on whether
\[ \frac{\beta}{1+\beta} \psi (R/\alpha)(1-\psi)^{(1-\alpha)/\alpha}\chi^{(1-\alpha)/\alpha}  \]
Is larger than or smaller than 1, either E-firms or F-firms will stop operating in the long run (respectively).

\problempart{(9)}
All labor goes to the E-firms after the F-firms stop operating. Hence
\[ Y_t = Y_{Et} = \chi^{1-\alpha} K_{Et}^\alpha N_{Et}^{1-\alpha} = \chi^{1-\alpha} K_{Et}^\alpha \]
So the law of motion of $K_{Et}$ is
\[ K_{E, t+1} = \kappa s_{E, t} = \frac{\beta}{1+\beta}\kappa m_t \]
\[ = \frac{\beta}{1+\beta}\psi Y_t = \frac{\beta}{1+\beta}\psi \chi^{1-\alpha} K_{Et}^\alpha \]
Note that since $\alpha < 1$, this no longer enjoys sustained growth, and will converge to a steady state at a higher level of capital, and wages are:
\[ w = (1-\alpha)\chi^{1-\alpha}K_{Et}^{\alpha} \]
and hence also converge to a higher level. 
\problempart{(10)}
In 6(ii) we showed the growth rate of $K_{Et}$ is
\[\frac{\beta}{1+\beta} \psi (R/\alpha)(1-\psi)^{(1-\alpha)/\alpha}\chi^{(1-\alpha)/\alpha}  \]
which is linearly proportional to $R$. Since $Y_t$ is an affine function of $K_{Et}$, and the coefficient of $K_{Et}$ in $Y_t$ is linearly proportional to $R$, $Y_t$ will grow faster at higher $R$ (even though the constant term is slightly decreased at higher $R$). From the expression from part (1) the normal worker's wages are pinned down by the F-firm's FOC, and are inversely proportional to $R$. So wages are lower at higher $R$, but growth rate is higher at higher $R$.
\problempart{(11)}
If E-firms can borrow at rental rate $R$, then they aren't constrained by their supply of capital. Therefore, because $\chi^{1-\alpha} > 1$, the E-firms immediately offer a higher wage than any F-firm can match, so all the workers just work for the E-firms, no F-firms operate, and we reach steady state.
\pagebreak
\problem{2}

\problempart{(1)}
From the firm FOC: wage is
\[ w = \sqrt{k} \]
Utility is Cobb-Douglas variant, so the consumer saves $(1/3)/(4/3) = 1/4$ of his/her income, so savings is
\[ s = \sqrt{k} / 4\]
\problempart{(2)}
Noting the probability $p$ given in the problem is $1/2$, we have the investment decision is to maximize:
\[ \frac{1}{2} \log (4s_R + (s-s_R) )+ \frac{1}{2}\log(s - s_R) \]
\[ \frac{1}{2} \log (3s_R + s) )+ \frac{1}{2}\log(s - s_R) \]
where $s_R$ is the savings in the risky asset. The FOCs above are
\[ \frac{3}{2(3s_R + s)} = \frac{1}{2(s - s_R)}\]
\[ 3(s- s_R) = 3s_R + s \]
\[ 2s = 6s_R \]
\[ s_R = \frac{1}{3}s \]
So the consumer puts 1/3 of their savings into the risky asset, 2/3 into the safe asset.
\problempart{(3)}
If the risky investment is successful, \[ k_{t+1} = 4((1/3)\sqrt{k_t}/4) + (2/3) (\sqrt{k_t}/4) = \frac{\sqrt{k_t}}{2}  \]
In the risky investment is not successful, \[ k_{t+1} = (2/3) (\sqrt{k_t}/4) = \frac{\sqrt{k_t}}{6}  \]
\problempart{(4)}
The stochastic steady state is in between the steady-states if the investment is always successful and if the investment is always failed: that is,
\[ k^* \in \left[\frac{1}{36}, \frac{1}{4} \right] \]
For a geometric representation, we plot the phase diagram, and demonstrate an example of the evolution of capital after a period of success and then a period without investment success. Capital initially increases towards the high steady state after the success, but falls towards the low steady state after failure.

\begin{tikzpicture}
\draw[thick,->] (0,0) -- (9,0) node[anchor=north west] {$k$};
\draw[thick,->] (0,0) -- (0,9) node[anchor=south east] {$k'$};
\draw (0,0) -- (8,8) node[anchor=south west] {$k=k'$};
\draw[blue] plot [smooth] coordinates {(0,0) (1, 2) (4,4) (9,6) } node[anchor=south west] {Successful investment};
\draw[red] plot [smooth] coordinates {(0,0) (1, 1) (4,2) (9,3) } node[anchor=south west] {Unsuccessful investment};
\draw[dashed] (2, 2.8) -- (2, 0) node[anchor=north] {$k_0$};
\draw[thick,->] (2,2.8) -- (2.8,2.8);
\draw[dashed] (2.8, 2.8) -- (2.8, 0) node[anchor=north] {$k_1$};
\draw[thick, ->] (2.8,1.7) -- (1.7, 1.7);
\draw[dashed] (1.7,1.7) -- (1.7, 0) node[anchor=north east] {$k_2$};
\end{tikzpicture}
\problempart{(5)}
In the good state, $k_1^H = 0.5 \sqrt{0.1}$. Consumption in period 1 is then $3/4$
\problempart{(6)}
\problempart{(7)}
\pagebreak
\problem{3}

\problempart{(1)}
\problempart{(2)}
\problempart{(3)}
\pagebreak
\problem{3}

\problempart{(1)}
\problempart{(2)}
\problempart{(3)}
\problempart{(4)}
\end{document}
	% line of code telling latex that your document is ending. If you leave this out, you'll get an error
