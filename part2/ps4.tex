%Jennifer Pan, August 2011

\documentclass[10pt,letter]{article}
	% basic article document class
	% use percent signs to make comments to yourself -- they will not show up.

\usepackage{amsmath}
\usepackage{amssymb}
\usepackage{tikz}
	% packages that allow mathematical formatting

\usepackage{graphicx}
	% package that allows you to include graphics

\usepackage{setspace}
	% package that allows you to change spacing

\onehalfspacing
	% text become 1.5 spaced

\usepackage{fullpage}
	% package that specifies normal margins

\renewcommand{\vector}[1]{\boldsymbol{#1}}
\newcommand{\problem}[1]{\section*{Problem #1}}
\newcommand{\problempart}[1]{\paragraph{#1}}

\begin{document}
	% line of code telling latex that your document is beginning


\title{Problem Set 4}

\author{Nicholas Wu}

\date{Fall 2020}
	% Note: when you omit this command, the current dateis automatically included

\maketitle
	% tells latex to follow your header (e.g., title, author) commands.
\textbf{Note:} I use bold symbols to denote vectors and nonbolded symbols to denote scalars. I primarily use vector notation to shorthand some of the sums, since many of the sums are dot products.

\problem{1}

\problempart{(1)}
\problempart{(2)}
\problempart{(3)}
\problempart{(4)}
\problempart{(5)}
\problempart{(6)}
\problempart{(7)}
\problempart{(8)}
\problempart{(9)}
\problempart{(10)}
\problempart{(11)}
\pagebreak
\problem{2}

\problempart{(1)}
\problempart{(2)}
\problempart{(3)}
\problempart{(4)}
\problempart{(5)}
\problempart{(6)}
\problempart{(7)}
\pagebreak
\problem{3}

\problempart{(1)}
\problempart{(2)}
\problempart{(3)}
\pagebreak
\problem{3}

\problempart{(1)}
\problempart{(2)}
\problempart{(3)}
\problempart{(4)}
\end{document}
	% line of code telling latex that your document is ending. If you leave this out, you'll get an error
