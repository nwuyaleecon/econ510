%Jennifer Pan, August 2011

\documentclass[10pt,letter]{article}
	% basic article document class
	% use percent signs to make comments to yourself -- they will not show up.

\usepackage{amsmath}
\usepackage{amssymb}
\usepackage{tikz}
\usepackage{enumitem}
	% packages that allow mathematical formatting

\usepackage{graphicx}
	% package that allows you to include graphics

\usepackage{setspace}
	% package that allows you to change spacing

\onehalfspacing
	% text become 1.5 spaced

\usepackage{fullpage}
	% package that specifies normal margins

\renewcommand{\vector}[1]{\boldsymbol{#1}}
\newcommand{\problem}[1]{\section*{Problem #1}}
\newcommand{\problempart}[1]{\paragraph{#1}}

\begin{document}
	% line of code telling latex that your document is beginning


\title{Problem Set 4}

\author{Nicholas Wu}

\date{Fall 2020}
	% Note: when you omit this command, the current dateis automatically included

\maketitle
	% tells latex to follow your header (e.g., title, author) commands.
\textbf{Note:} I use bold symbols to denote vectors and nonbolded symbols to denote scalars. I primarily use vector notation to shorthand some of the sums, since many of the sums are dot products.

\problem{1}

\problempart{(1)}
The F-firms are the normal type we have seen:
\[ R = \alpha k_F^{\alpha - 1}n_F^{1-\alpha} \]
\[ w = (1- \alpha) k_F^{\alpha}n_F^{-\alpha} \]
Since the measure of the F firms is 1
\[ \frac{K_F}{N_F} = \left(\frac{\alpha}{R}\right)^{1/(1-\alpha )} \]
\[ \frac{K_F}{Y} = \frac{k_f}{k_F^{\alpha}n_F^{1-\alpha}} = \frac{1}{k_F^{\alpha-1}n_F^{1-\alpha}} = \frac{\alpha}{R} \]
\problempart{(2)}
We already showed the equilibrium wage for the $F$-firm above:
\[ w = (1- \alpha) k_F^{\alpha}n_F^{-\alpha} = (1-\alpha) \left(\frac{\alpha}{R} \right)^{\alpha/(1-\alpha)}\]
Since this wage is competitive, this also must satisfy the wage FOC for maximization of the E-firm:
\[ w = (1-\alpha)(1-\psi)\chi^{1-\alpha}k_{E}^\alpha n_E^{-\alpha}  \]
\[ = (1-\alpha)(1-\psi)\chi^{1-\alpha}(\kappa s_{E})^\alpha (\kappa n_E)^{-\alpha}  \]
\[ = (1-\alpha)(1-\psi)\chi^{1-\alpha}S_E^\alpha N_E^{-\alpha}  \]
Using the wage expression, we get
\[ N_E^\alpha = \frac{(1-\alpha)(1-\psi)\chi^{1-\alpha}S_E^\alpha}{(1-\alpha) \left(\frac{\alpha}{R} \right)^{\alpha/(1-\alpha)}} = \frac{(1-\psi)\chi^{1-\alpha}S_E^\alpha}{\left(\alpha/R \right)^{\alpha/(1-\alpha)}} \]
So
\[ N_E = \frac{(1-\psi)^{1/\alpha}\chi^{(1-\alpha)/\alpha}S_E}{\left(\alpha/R \right)^{1/(1-\alpha)}} \]
\[ N_F = 1 - \frac{(1-\psi)^{1/\alpha}\chi^{(1-\alpha)/\alpha}S_E}{\left(\alpha/R \right)^{1/(1-\alpha)}} \]
\[ m = \psi S_E^\alpha \chi^{1-\alpha} N_E^{1-\alpha}\kappa^{-1}\]
\[ = \psi S_E^\alpha \chi^{1-\alpha} \frac{R(1-\psi)^{(1-\alpha)/\alpha}\chi^{(1-\alpha)^2/\alpha}S_E^{1-\alpha}}{\alpha \kappa} \]
\[ = \frac{R\psi (1-\psi)^{(1-\alpha)/\alpha}\chi^{(1-\alpha)/\alpha}S_E}{\alpha \kappa} \]
\problempart{(3)}
The profit is
\[ (1-\psi)\chi^{1-\alpha}s_{E}^\alpha n_E^{1-\alpha} - wn_E \]
\[ = (1-\psi)\chi^{1-\alpha}S_{E}^\alpha N_E^{1-\alpha}\kappa^{-1} - w\kappa^{-1}N_E \]
\[ = (1-\psi)\chi^{1-\alpha}S_{E}^\alpha \frac{(1-\psi)^{(1-\alpha)/\alpha}\chi^{(1-\alpha)^2/\alpha}S_E^{1-\alpha}}{\alpha/R }\kappa^{-1} - (1-\alpha) \left(\frac{\alpha}{R} \right)^{\alpha/(1-\alpha)}\kappa^{-1}\frac{(1-\psi)^{1/\alpha}\chi^{(1-\alpha)/\alpha}S_E}{\left(\alpha/R \right)^{1/(1-\alpha)}} \]
\[ = RS_{E} \frac{(1-\psi)^{1/\alpha}\chi^{(1-\alpha)/\alpha}}{\alpha\kappa } - (1-\alpha) R \frac{(1-\psi)^{1/\alpha}\chi^{(1-\alpha)/\alpha}S_E}{\alpha\kappa  } \]
\[ = RS_{E} \frac{(1-\psi)^{1/\alpha}\chi^{(1-\alpha)/\alpha}}{\kappa }  \]
\[ = Rs_{E}(1-\psi)^{1/\alpha}\chi^{(1-\alpha)/\alpha}  \]
So the return rate on savings $s_E$ is
\[ R(1-\psi)^{1/\alpha}\chi^{(1-\alpha)/\alpha} \]
\problempart{(4)}
The utility is equivalent to Cobb-Douglas between the two period goods, so each type of consumer saves $\beta/(1+\beta)$ fraction of their income.
\[ s_W = \frac{\beta}{1+\beta} w \]
\[ s_E = \frac{\beta}{1+\beta} m \]
\problempart{(5)}
We need the coefficient of $R$ for the return rate on saavings from part (3) to be greater than 1. That is,
\[(1-\psi)^{1/\alpha}\chi^{(1-\alpha)/\alpha} > 1\]
taking the $\alpha$ power of both sides
\[ (1-\psi) \chi^{1-\alpha} > 1\]
\[ \chi^{1-\alpha} > \frac{1}{1-\psi} \]
The LHS is the productivity factor of the E-firm, so we need this productivity to offset the cost of paying the manager using the firm profit. If this condition fails, no one will invest their savings into E-firms, and hence no E-firms will operate after the next period because they will have no capital.
\problempart{(6)}

\begin{enumerate}[label=(\roman*)]
\item We first find $N_{Et}, N_{Ft}$:
\[ N_{Et} = \frac{(1-\psi)^{1/\alpha}\chi^{(1-\alpha)/\alpha}S_E}{\left(\alpha/R \right)^{1/(1-\alpha)}} = \frac{(1-\psi)^{1/\alpha}\chi^{(1-\alpha)/\alpha}K_{Et}}{\left(\alpha/R \right)^{1/(1-\alpha)}} \]
\[ N_{Ft} = 1 - \frac{(1-\psi)^{1/\alpha}\chi^{(1-\alpha)/\alpha}K_{Et}}{\left(\alpha/R \right)^{1/(1-\alpha)}} \]
Then
\[ Y_{Et} = \chi^{1-\alpha} K_{Et}^\alpha N_{Et}^{1-\alpha} \]
\[ =  (R/\alpha)(1-\psi)^{(1-\alpha)/\alpha}\chi^{(1-\alpha)/\alpha}K_{Et}\]
And from part (1):
\[ \frac{K_{Ft}}{N_{Ft}} = \left(\frac{\alpha}{R}\right)^{1/(1-\alpha )} \]
\[ K_{Ft} = \left(\frac{\alpha}{R}\right)^{1/(1-\alpha )} N_{Ft} \]
\[ Y_{Ft} = \left(\frac{\alpha}{R}\right)^{\alpha/(1-\alpha)} N_{Ft} \]
\[ = \left(\frac{\alpha}{R}\right)^{\alpha/(1-\alpha)} \left(1 - \frac{(1-\psi)^{1/\alpha}\chi^{(1-\alpha)/\alpha}K_{Et}}{\left(\alpha/R \right)^{1/(1-\alpha)}} \right) \]
So total output
\[ Y_t = Y_{Et} + Y_{Ft} = (R/\alpha)(1-\psi)^{(1-\alpha)/\alpha}\chi^{(1-\alpha)/\alpha}K_{Et} + \left(\frac{\alpha}{R}\right)^{\alpha/(1-\alpha)} \left(1 - \frac{(1-\psi)^{1/\alpha}\chi^{(1-\alpha)/\alpha}K_{Et}}{\left(\alpha/R \right)^{1/(1-\alpha)}} \right) \]
\[ = (R/\alpha)(1-\psi)^{(1-\alpha)/\alpha}\chi^{(1-\alpha)/\alpha}K_{Et} + \left(\frac{\alpha}{R}\right)^{\alpha/(1-\alpha)}  - (R/\alpha)(1-\psi)^{1/\alpha}\chi^{(1-\alpha)/\alpha}K_{Et}  \]
\[ = (R/\alpha)\chi^{(1-\alpha)/\alpha}K_{Et} \left((1-\psi)^{(1-\alpha)/\alpha} - (1-\psi)^{1/\alpha} \right) + \left(\frac{\alpha}{R}\right)^{\alpha/(1-\alpha)}   \]
\[ = (R/\alpha)\psi(1-\psi)^{(1-\alpha)/\alpha}\chi^{(1-\alpha)/\alpha}K_{Et}  + \left(\frac{\alpha}{R}\right)^{\alpha/(1-\alpha)}   \]
\item The law of motion for $K_{Et}$:
\[ K_{E,t+1} = \kappa s_E = \kappa \frac{\beta}{1+\beta} m \]
\[ =  \frac{\beta}{1+\beta} \psi K_{Et}^\alpha \chi^{1-\alpha} N_{Et}^{1-\alpha} \]
\[ = \frac{\beta}{1+\beta} \psi (R/\alpha)(1-\psi)^{(1-\alpha)/\alpha}\chi^{(1-\alpha)/\alpha}K_{Et} \]
\item Finally, we can slog through the algebra to compute out $\rho_t$:
\[ \rho_t = \frac{K_{Et}}{K_{Et} + K_{Ft}}\rho_E + \frac{K_{Ft}}{K_{Et} + K_{Ft}}\rho_F \]
\[ = \frac{K_{Et}}{K_{Et} + K_{Ft}}(\alpha\chi^{1-\alpha}K_{Et}^{\alpha-1}N_{Et}^{1-\alpha}) + \frac{K_{Ft}}{K_{Et} + K_{Ft}}(\alpha K_{Ft}^{\alpha-1}N_{Ft}^{1-\alpha}) \]
\[ = \frac{\alpha Y_{Et}}{K_{Et} + K_{Ft}} + \frac{\alpha Y_{Ft}}{K_{Et} + K_{Ft}} \]
\[ = \frac{\alpha Y_{t}}{K_{Et} + K_{Ft}}  \]
\[ = \frac{R\psi(1-\psi)^{(1-\alpha)/\alpha}\chi^{(1-\alpha)/\alpha}K_{Et}  + \alpha \left(\frac{\alpha}{R}\right)^{\alpha/(1-\alpha)} }{K_{Et} + \left(\frac{\alpha}{R}\right)^{1/(1-\alpha )} N_{Ft}}  \]
\[ = \frac{R\psi(1-\psi)^{(1-\alpha)/\alpha}\chi^{(1-\alpha)/\alpha}K_{Et}  + \alpha \left(\frac{\alpha}{R}\right)^{\alpha/(1-\alpha)} }{K_{Et} + \left(\frac{\alpha}{R}\right)^{1/(1-\alpha )} \left(1 - \frac{(1-\psi)^{1/\alpha}\chi^{(1-\alpha)/\alpha}K_{Et}}{\left(\alpha/R \right)^{1/(1-\alpha)}} \right)}  \]
\[ = \frac{R\psi(1-\psi)^{(1-\alpha)/\alpha}\chi^{(1-\alpha)/\alpha}K_{Et}  + \alpha \left(\frac{\alpha}{R}\right)^{\alpha/(1-\alpha)} }{K_{Et} + \left(\frac{\alpha}{R}\right)^{1/(1-\alpha )} - (1-\psi)^{1/\alpha}\chi^{(1-\alpha)/\alpha}K_{Et}}  \]
\[ = \frac{R\psi(1-\psi)^{(1-\alpha)/\alpha}\chi^{(1-\alpha)/\alpha}K_{Et}  + \alpha \left(\frac{\alpha}{R}\right)^{\alpha/(1-\alpha)} }{K_{Et} + \left(\frac{\alpha}{R}\right)^{1/(1-\alpha )} - (1-\psi)^{1/\alpha}\chi^{(1-\alpha)/\alpha}K_{Et}}  \]
\end{enumerate}
\problempart{(7)}
The wages for normal workers is
\[ w = (1-\alpha) \left(\frac{\alpha}{R} \right)^{\alpha/(1-\alpha)} \]
from the firm FOC. This is thus constant over time, since the FOCs for the F-firms fully determines this wage.
\problempart{(8)}
From part (6), we see the law of motion for $K_{Et}$ can be rewritten:
\[ \frac{K_{E,t+1}}{K_{E,t}} = \frac{\beta}{1+\beta} \psi (R/\alpha)(1-\psi)^{(1-\alpha)/\alpha}\chi^{(1-\alpha)/\alpha} \]
Thus $K_{E}$ has constant growth rate. If the RHS is $< 1$, then the stock of E-firm capital must decrease to 0, and in the long run no E-firms will operate. If the RHS is $>1$, then the stock of E-firm capital is growing, and since
\[ N_{Et} =  \frac{(1-\psi)^{1/\alpha}\chi^{(1-\alpha)/\alpha}}{\left(\alpha/R \right)^{1/(1-\alpha)}}K_{Et} \]
Since $N_{Et}$ is linear in $K_{Et}$ which grows at a constant rate, eventually $N_{Et} = 1$ due to the labor constraint. This implies $N_{Ft} = 0$ eventually, and since no workers work for F-firms, all the F-firms stop operating. Therefore, depending on whether
\[ \frac{\beta}{1+\beta} \psi (R/\alpha)(1-\psi)^{(1-\alpha)/\alpha}\chi^{(1-\alpha)/\alpha}  \]
Is larger than or smaller than 1, either E-firms or F-firms will stop operating in the long run (respectively).

\problempart{(9)}
All labor goes to the E-firms after the F-firms stop operating. Hence
\[ Y_t = Y_{Et} = \chi^{1-\alpha} K_{Et}^\alpha N_{Et}^{1-\alpha} = \chi^{1-\alpha} K_{Et}^\alpha \]
So the law of motion of $K_{Et}$ is
\[ K_{E, t+1} = \kappa s_{E, t} = \frac{\beta}{1+\beta}\kappa m_t \]
\[ = \frac{\beta}{1+\beta}\psi Y_t = \frac{\beta}{1+\beta}\psi \chi^{1-\alpha} K_{Et}^\alpha \]
Note that since $\alpha < 1$, this no longer enjoys sustained growth, and will converge to a steady state at a higher level of capital, and wages are:
\[ w = (1-\alpha)\chi^{1-\alpha}K_{Et}^{\alpha} \]
and hence also converge to a higher level.
\problempart{(10)}
In 6(ii) we showed the growth rate of $K_{Et}$ is
\[\frac{\beta}{1+\beta} \psi (R/\alpha)(1-\psi)^{(1-\alpha)/\alpha}\chi^{(1-\alpha)/\alpha}  \]
which is linearly proportional to $R$. Since $Y_t$ is an affine function of $K_{Et}$, and the coefficient of $K_{Et}$ in $Y_t$ is linearly proportional to $R$, $Y_t$ will grow faster at higher $R$ (even though the constant term is slightly decreased at higher $R$). From the expression from part (1) the normal worker's wages are pinned down by the F-firm's FOC, and are inversely proportional to $R$. So wages are lower at higher $R$, but growth rate is higher at higher $R$.
\problempart{(11)}
If E-firms can borrow at rental rate $R$, then they aren't constrained by their supply of capital. Therefore, because $\chi^{1-\alpha} > 1$, the E-firms immediately offer a higher wage than any F-firm can match, so all the workers just work for the E-firms, no F-firms operate, and we reach steady state.
\pagebreak
\problem{2}

\problempart{(1)}
From the firm FOC: wage is
\[ w = \sqrt{k} \]
Utility is Cobb-Douglas variant, so the consumer saves $(1/3)/(4/3) = 1/4$ of his/her income, so savings is
\[ s = \sqrt{k} / 4\]
\problempart{(2)}
Noting the probability $p$ given in the problem is $1/2$, we have the investment decision is to maximize:
\[ \frac{1}{2} \log (4s_R + (s-s_R) )+ \frac{1}{2}\log(s - s_R) \]
\[ \frac{1}{2} \log (3s_R + s) )+ \frac{1}{2}\log(s - s_R) \]
where $s_R$ is the savings in the risky asset. The FOCs above are
\[ \frac{3}{2(3s_R + s)} = \frac{1}{2(s - s_R)}\]
\[ 3(s- s_R) = 3s_R + s \]
\[ 2s = 6s_R \]
\[ s_R = \frac{1}{3}s \]
So the consumer puts 1/3 of their savings into the risky asset, 2/3 into the safe asset.
\problempart{(3)}
If the risky investment is successful, \[ k_{t+1} = 4((1/3)\sqrt{k_t}/4) + (2/3) (\sqrt{k_t}/4) = \frac{\sqrt{k_t}}{2}  \]
In the risky investment is not successful, \[ k_{t+1} = (2/3) (\sqrt{k_t}/4) = \frac{\sqrt{k_t}}{6}  \]
\problempart{(4)}
The stochastic steady state is in between the steady-states if the investment is always successful and if the investment is always failed: that is,
\[ k^* \in \left[\frac{1}{36}, \frac{1}{4} \right] \]
For a geometric representation, we plot the phase diagram, and demonstrate an example of the evolution of capital after a period of success and then a period without investment success. Capital initially increases towards the high steady state after the success, but falls towards the low steady state after failure.

\begin{tikzpicture}
\draw[thick,->] (0,0) -- (9,0) node[anchor=north west] {$k$};
\draw[thick,->] (0,0) -- (0,9) node[anchor=south east] {$k'$};
\draw (0,0) -- (8,8) node[anchor=south west] {$k=k'$};
\draw[blue] plot [smooth] coordinates {(0,0) (1, 2) (4,4) (9,6) } node[anchor=south west] {Successful investment};
\draw[red] plot [smooth] coordinates {(0,0) (1, 1) (4,2) (9,3) } node[anchor=south west] {Unsuccessful investment};
\draw[dashed] (2, 2.8) -- (2, 0) node[anchor=north] {$k_0$};
\draw[thick,->] (2,2.8) -- (2.8,2.8);
\draw[dashed] (2.8, 2.8) -- (2.8, 0) node[anchor=north] {$k_1$};
\draw[thick, ->] (2.8,1.7) -- (1.7, 1.7);
\draw[dashed] (1.7,1.7) -- (1.7, 0) node[anchor=north east] {$k_2$};
\end{tikzpicture}
\problempart{(5)}
The $t=0$ generation will have received wage $w = \sqrt{k_0}$. They invest $\sqrt{k_0}/4$, $1/3$ of which goes into the risky asset, $2/3$ of which stay in the safe asset. So their consumption in the good state will be
\[ 4(1/3)\frac{\sqrt{k_0}}{4} + (2/3)\frac{\sqrt{k_0}}{4} = \sqrt{k_0}/2 \approx 0.1581 \]
and in the bad state:
\[ (2/3)\frac{\sqrt{k_0}}{4} = \sqrt{k_0}/6 \approx 0.0527 \]
\problempart{(6)}
The tax falls on the consumers, so the equilibrium wage does not change since it is determined by the firm's FOC. The consumer's income is then halved:
\[ I = \sqrt{k}/2 \]
The utility is still a Cobb-Douglas variant, so the consumer still invests $1/4$ of their income.
\[ s = \sqrt{k}/8 \]
The assets have not changed, so the consumer still puts $1/3$ of their savings into the risky asset. So the laws of motion can be determined.
If the risky investment is successful, \[ k_{t+1} = 4((1/3)\sqrt{k_t}/8) + (2/3) (\sqrt{k_t}/8) = \frac{\sqrt{k_t}}{4}  \]
In the risky investment is not successful, \[ k_{t+1} = (2/3) (\sqrt{k_t}/8) = \frac{\sqrt{k_t}}{12}  \]
Then the new range of steaady states is:
\[ (k^*)' = \left[\frac{1}{144}, \frac{1}{16} \right] \]
The steady state capital levels fall by a factor of 4, so steady state output is halved.
\problempart{(7)}
This is not a great policy. The tax drives investment down, and reduces the payment to the elderly as time passes. However, it is not Pareto inferior- the first generation, whose tax payments were harvested at higher capital levels, will be better off, so at least one generation is better off under this policy.
\pagebreak
\problem{3}

\problempart{(1)}
\begin{itemize}
  \item The final good producer problem is given by
  \[ \max AL^{1-\alpha} \int_0^{N_t}(x_{j}^\alpha) \ dj - wL - \int_0^{N_t} p_j x_{j} \ dj \]
  The FOC on labor gives
  \[ w = (1-\alpha)AL^{-\alpha) }\int_0^{N_t}(x_{j}^\alpha)  \]
  The FOC on $x_{ji}$ gives
  \[ p_j = \alpha AL^{1-\alpha}x_{j}^{\alpha - 1} \]
  \item The monopolist problem is
  \[ \max \ (\alpha AL^{1-\alpha}x_{j}^{\alpha - 1}) x_{j} - 1 x_{j} \]
  \[ \max \ (\alpha AL^{1-\alpha}x_{j}^{\alpha}) - 1 x_{j} \]
  This has the FOC:
  \[ \alpha^2 AL^{1-\alpha} x_j^{1-\alpha} = 1 \]
  \[ \alpha p_j = 1 \]
  \[ p_j = \alpha^{-1} \]
  So
  \[ x_{jt} = \alpha^{2/(1-\alpha)}A^{1/(1-\alpha)}L \]
  \[ \pi_{jt} = (1-\alpha)\alpha^{(1+\alpha)/(1-\alpha)}A^{1/(1-\alpha)}L \]
  \item The value is given by the Bellman equation:
  \[ r_t V_{jt} - \dot{V}_{jt} = \pi_{jt}  \]
  \item The free entry condition requires $\beta V_j = 1 $.
  \item Then
   \[ r_t/\beta - \dot{V}_{jt} = (1-\alpha)\alpha^{(1+\alpha)/(1-\alpha)}A^{1/(1-\alpha)}L  \]
    \[ r_t = \beta(1-\alpha)\alpha^{(1+\alpha)/(1-\alpha)}A^{1/(1-\alpha)}L  \]
\end{itemize}
\problempart{(2)}
The EE from consumption gives us
\[ \dot{c}/c = \frac{1}{\theta}(r_t-\rho) = g \]

Output is
\[ Y_i = AL_i^{1-\alpha} N x_i^\alpha \]
Net output of the final good is given by
\[ \int_i Y_i = AN \int_i L_i^{1-\alpha} x_i^\alpha \]
\[ = AN \int_i L_i (x_i/L_i)^\alpha \]
Note that $x_i/L_i$ is constant for all firms, since they have the same production function. Hence
\[ = AN \left(x/L\right)^\alpha L \]
\[ = AN x^\alpha L^{1-\alpha} \]
So net output of the final good also grows at the same rate as $N$ and hence as $c$, which is the rate $g$. For $\dot{N} > 0$, we require
\[ \frac{1}{\theta}(r_t-\rho) > 0\]
\[ r_t > \rho \]
\[ \beta(1-\alpha)\alpha^{(1+\alpha)/(1-\alpha)}A^{1/(1-\alpha)}L  > \rho \]
Transversality condition also gives
\[ (1-\theta)g < \rho \]
Plugging in for $r_t$ the growth rate is
\[ g = \frac{1}{\theta}\left(\beta(1-\alpha)\alpha^{(1+\alpha)/(1-\alpha)}A^{1/(1-\alpha)}L-\rho\right) \]
\problempart{(3)}
\begin{itemize}
  \item We need to determine the new interest rate $r'_t$. The altered
  \[ x'_{jt} = \alpha^{2/(1-\alpha)}A^{1/(1-\alpha)}(1-s)L \]
  \[ \pi'_{jt} = (1-\alpha)\alpha^{(1+\alpha)/(1-\alpha)}A^{1/(1-\alpha)}(1-s)L \]
  Applying free-entry to the value function again, we get
  \[ r'_t/\beta = \pi'  \]
  \[ r'_t = \beta (1-\alpha)\alpha^{(1+\alpha)/(1-\alpha)}A^{1/(1-\alpha)}(1-s)L \]
  So the new growth rate is
  \[ g' = \frac{1}{\theta}\left(  \beta (1-\alpha)\alpha^{(1+\alpha)/(1-\alpha)}A^{1/(1-\alpha)}(1-s)L  - \rho \right) \]

  \item Higher $s$ decreases $(1-s)$, which in turn reduces $g'$. This is because higher $s$ reduces the labor force available to the normal non-government firms, and they will do less R\&D and hence the economy growth rate is slower.
  \item The first type of social services might increase overall productivity through $A$, the second type might increase $\beta$ or the effectiveness of research. If $s$ can increase $A$ and $\beta$, then it is possible that higher $s$ can result in sufficient increases in both of these to offset the $(1-s)$ factor. There is room for Pareto improvement in the pricing of $x_j$ goods. Since these are monopolistically priced, the prices are too high and there is inefficiently low usage of the intermediate goods.
\end{itemize}
\pagebreak
\problem{3}

\problempart{(1)}
The final good producer FOC in $x_j$ gives
\[ p_j = \alpha x_j^{\alpha - 1}L^{1-\alpha} = \alpha x_j^{\alpha - 1} \]
Then the monopolistic intermediate good producers satisfy
\[ \alpha^2 x_j^{\alpha - 1} = 1 \]
\[ p_j = 1/\alpha \]
So
\[ x_j = \alpha^{2/(1-\alpha)} \]
\[ \pi_j = (p_j - 1) \alpha^{2/(1-\alpha)} = (1-\alpha)\alpha^{(1+\alpha)/(1-\alpha)} \]
Then aggregate output is
\[ Y = A x_j^\alpha = A \alpha^{2\alpha/(1-\alpha)}\]
\[ X = \int_0^A x_j = A\alpha^{2/(1-\alpha)} \]
Assuming investment in research $Z$ is determined,
\[ C = Y - X - Z\]
\problempart{(2)}
The Bellman equation on $V$ is
\[ r_t V - \dot{V} = \pi \]
If there is free entry, $V = \mu$ (the cost of invention). So
\[ r_t \mu = \pi \]
\[ r_t = \mu^{-1}(1-\alpha)\alpha^{(1+\alpha)/(1-\alpha)} \]
The growth rate is then
\[ g = \frac{1}{\theta}(r_t - \rho) = \theta^{-1}(\mu^{-1}(1-\alpha)\alpha^{(1+\alpha)/(1-\alpha)} - \rho) \]
The output growth rate grows at the same rate as consumption, so this is the growth rate of the economy. Then $\mu Z/ A = g$, and assuming $g > 0$ and transversality, this is the equilibrium growth rate.
\problempart{(3)}
Because the monopolist can lose their patent at rate $\xi$, the Bellman equation becomes
\[ (r + \xi)V - \dot{V} = \pi \]
The patent expiry does not affect the monopoly pricing, so $\pi$ is still unchanged, and hence we have
\[ r + \xi = \mu^{-1}(1-\alpha)\alpha^{(1+\alpha)/(1-\alpha)} \]
\[ r = \mu^{-1}(1-\alpha)\alpha^{(1+\alpha)/(1-\alpha)} - \xi \]
The growth rate is then
\[ g = \frac{1}{\theta}(r - \rho) = \theta^{-1}(\mu^{-1}(1-\alpha)\alpha^{(1+\alpha)/(1-\alpha)} - \xi - \rho)\]
Competitive firms charge price $1$, (no profits), so the quantity demanded is $x_j = \alpha^{1/(1-\alpha)}$. Let the measure of monopolistic intermediate good firms be $A_m$, and competitive intermeidate goods be $A_c$. Then we have

\[ Y = A_m x_{jm}^\alpha + A_c x_{jc}^\alpha = A_m \alpha^{2\alpha/(1-\alpha)} + A_c \alpha^{\alpha/(1-\alpha)}\]
\[ X = \int_0^A x_j = A_m \alpha^{2/(1-\alpha)} + A_c \alpha^{1/(1-\alpha)}  \]

\problempart{(4)}
Yes; because all current intermediate goods have no patents, the prices of these goods will be competitively determined and the final good production will be efficient. At the same time, because the government enforces patents going forward, the growth rate in the monopolistic patent case will be maintained because those patents should be respected, and hence the economy will also retain the same rate of innovation. In practice, this may be difficult to believe because doing so may erode trust in the government's enforcement of patents; i.e., inventors may be distrustful of the government's enforcement of patents in the future.
\end{document}
	% line of code telling latex that your document is ending. If you leave this out, you'll get an error
