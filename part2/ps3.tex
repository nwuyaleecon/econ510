%Jennifer Pan, August 2011

\documentclass[10pt,letter]{article}
	% basic article document class
	% use percent signs to make comments to yourself -- they will not show up.

\usepackage{amsmath}
\usepackage{amssymb}
\usepackage{tikz}
	% packages that allow mathematical formatting

\usepackage{graphicx}
	% package that allows you to include graphics

\usepackage{setspace}
	% package that allows you to change spacing

\onehalfspacing
	% text become 1.5 spaced

\usepackage{fullpage}
	% package that specifies normal margins

\renewcommand{\vector}[1]{\boldsymbol{#1}}
\newcommand{\problem}[1]{\section*{Problem #1}}
\newcommand{\problempart}[1]{\paragraph{#1}}

\begin{document}
	% line of code telling latex that your document is beginning


\title{Problem Set 3}

\author{Nicholas Wu}

\date{Fall 2020}
	% Note: when you omit this command, the current dateis automatically included

\maketitle
	% tells latex to follow your header (e.g., title, author) commands.
\textbf{Note:} I use bold symbols to denote vectors and nonbolded symbols to denote scalars. I primarily use vector notation to shorthand some of the sums, since many of the sums are dot products.

\problem{1}

\problempart{(1)}
The optimal time length of schooling maximizes:
\[ \max_S \int_S^\infty e^{-rt}w(t)h(t) \]
\[ \max_S \int_S^\infty e^{-(r - g)t}S^{\alpha} \]
\[ \max_S \frac{S^{\alpha}w(0)e^{-(r-g)S}}{r-g} \]
\[ \frac{\alpha S^{\alpha - 1}}{S^\alpha} = r-g\]
\[  S = \frac{\alpha}{r-g} = 11.25\]
\problempart{(2)}
The present value of consumption is
\[ \int_S^\infty e^{-rt}c_t = \int_S^\infty w(0) e^{-(r-g)t}S^\alpha \]
\[  = \frac{w(0) e^{-(r-g)S}S^\alpha }{r-g}\]
\problempart{(3)}
We need
\[ \max  \int e^{-(\rho + \nu)t} \log c(t) \]
subject to
\[ \int e^{- rt}c(t) = \frac{w(0) e^{-(r-g)S}S^\alpha }{r-g} \]
we have
\[ e^{-(\rho + \nu)t} \frac{1}{c(t)} = \lambda e^{-rt}\]
\[ \frac{1}{\lambda }e^{-(\rho + \nu - r)t}  = c(t) \]
and we get the EE:
\[ \frac{\dot{c}(t)}{c(t)}= r - \rho - \nu  \]
which implies
\[ c(t) = c(0) e^{(r-\rho-\nu)t} \]
Plugging into budget constraint:
\[ \frac{c(0)}{\rho+\nu} = \frac{w(0) e^{-(r-g)S}S^\alpha }{r-g} \]
\[ c(0) = \frac{w(0)(\rho+\nu) e^{-(r-g)S}S^\alpha }{r-g} \]

\problempart{(4)}
Intuitively, since consumption is smoothed across time, individuals will go into debt before they reach their optimal level of schooling. Specifically, at time $T < S$, we have the present value of an individual's asset holding is given by
\[ -e^{rT}\int_0^T e^{-rt}c(t) \]
which is clearly negative. However, after the individual finishes schooling, the present value of asset holding is
\[ e^{rT}\left(\int_S^T e^{-rt} W_t dt - \int_0^T e^{-rt}c_t \right) \]
\[ = e^{rT}\left(w(0)S^{\alpha} \int_S^T e^{-(r-g)t} dt - c(0)\int_0^T  e^{-(\rho+\nu)t} \right) \]
\[ = e^{rT}\left(\frac{w(0)S^{\alpha}}{r-g} \left( e^{-(r-g)T} - e^{-(r-g)S}  \right)- \frac{w(0)(\rho+\nu) e^{-(r-g)S}S^\alpha }{r-g} \left(\frac{e^{-(\rho+\nu)T} - 1}{\rho+\nu}  \right) \right) \]
\[ = \frac{w(0)S^{\alpha}e^{rT}}{r-g}\left( \left( e^{-(r-g)T} - e^{-(r-g)S}  \right)-  e^{-(r-g)S} \left(e^{-(\rho+\nu)T} - 1  \right) \right) \]
\[ = \frac{w(0)S^{\alpha}e^{rT}}{r-g}\left(  e^{-(r-g)T} - e^{-(r-g)S} \left(e^{-(\rho+\nu)T} \right) \right) \]
\[ = \frac{w(0)S^{\alpha}e^{rT}e^{-(r-g)S}}{r-g}\left(   e^{-(r-g)(T-S)}- e^{-(\rho+\nu)T}  \right) \]
Since $r-g > \rho+\nu$, this will eventually turn positive at
\[ e^{-(r-g)(T-S) + (\rho + \nu)T} = 1 \]
\[ (r-g)(T-S) = (\rho + \nu)T \]
\[ (r-g - \rho - \nu)T = (r-g)S \]
\[ T = \frac{(r-g)S}{(r-g - \rho - \nu)} \]
so after that point, the individual paid of their student loans and will begin accummulating wealth.

\problempart{(5)}
If we take the $\log$ of the wage rate, we find that
\[ \log W(t) = \log w(0) + \alpha \log S + gt \]
which we can use to empirically measure the schooling effectivness parameter $\alpha$.
\problempart{(6)}
Imposing a borrowing constraint can induce people to take less schooling than they would otherwise, since due to a borrowing constraint they would have to reduce consumption as a result. Alternatively, individuals might also have to instead work first, and take schooling later in their life. One potential way to alleviate this issue is to provide students a stipend, funded from wages. That way, consumers may be able to still achieve the same welfare since they receive some income during their schooling.

\problem{2}

\problempart{(1)}
\problempart{(2)}
\problempart{(3)}
\problempart{(4)}
\problempart{(5)}
\problempart{(6)}
\problempart{(7)}
\problempart{(8)}
\problempart{(9)}

\problem{3}

\problempart{(1)}
\problempart{(2)}
\problempart{(3)}
\problempart{(4)}
\problempart{(5)}
\problempart{(6)}

\problem{4}

\problempart{(1)}
\problempart{(2)}
\problempart{(3)}
\problempart{(4)}
\problempart{(5)}
\problempart{(6)}
\problempart{(7)}

\problem{5}

\problempart{(1)}
\problempart{(2)}
\problempart{(3)}
\problempart{(4)}
\problempart{(5)}
\end{document}
	% line of code telling latex that your document is ending. If you leave this out, you'll get an error
