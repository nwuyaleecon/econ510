%Jennifer Pan, August 2011

\documentclass[10pt,letter]{article}
	% basic article document class
	% use percent signs to make comments to yourself -- they will not show up.

\usepackage{amsmath}
\usepackage{amssymb}
	% packages that allow mathematical formatting

\usepackage{graphicx}
	% package that allows you to include graphics

\usepackage{setspace}
	% package that allows you to change spacing

\onehalfspacing
	% text become 1.5 spaced

\usepackage{fullpage}
	% package that specifies normal margins

\renewcommand{\vector}[1]{\boldsymbol{#1}}
\newcommand{\problem}[1]{\section*{Problem #1}}
\newcommand{\problempart}[1]{\paragraph{#1}}

\begin{document}
	% line of code telling latex that your document is beginning


\title{Problem Set 1}

\author{Nicholas Wu}

\date{Fall 2020}
	% Note: when you omit this command, the current dateis automatically included

\maketitle
	% tells latex to follow your header (e.g., title, author) commands.
Note: I use bold symbols to denote vectors and nonbolded symbols to denote scalars.

\problem{1}

\problempart{(1)} Markets are only open at period 0, and goods for all periods of time are traded. Consumers may trade with anyone from their own consumer group or the other consumer group. Given price vector $\vector{p} = \{p_0, p_1, p_2, ... \}$ agents of type $i$ seek to maximize their own utility:
\[ U(\vector{c^i}) = \sum_{t=0}^\infty \beta^t \log c^i_t \]
 subject to their budget constraint:
\[ \vector{p}\cdot \vector{c^i} \le \vector{p}\cdot \vector{w^i} \]
where $\vector{c^i}$ denotes the vector of allocations given to type $i$ across time, and $\vector{w^i}$ denotes the vector of endowments given to type $i$ across times. \\\\
An Arrow-Debreu equilibrium must additionally satisfy allocation feasibility. That is, for all $t \ge 0$:
\[ \sum_{i=1}^2 c^i_t \le \sum_{i=1}^2 w^i_t \]
We also require allocations to be nonnegative:
\[ c^i_t \ge 0 \]
\problempart{(2)}
An allocation $(\vector{c^1}, \vector{c^2} )$ Pareto dominates another allocation $(\vector{\tilde{c}^1},\vector{\tilde{c}^2})$ if $\forall i$:
\[ u_i(\vector{c^i}) \ge u_i(\vector{\tilde{c}^i}) \]
and for some $i$,
\[ u_i(\vector{c^i}) > u_i(\vector{\tilde{c}^i}) \]
An allocation is Pareto efficient if no other allocation Pareto dominates it. The planner maximizes the following:
\[ \alpha_1 \sum_{t=0}^\infty \beta^t \log c^1_t + \alpha_2 \sum_{t=0}^\infty \beta^t \log c^2_t  \]
subject to ($\forall t$)
\[ c^1_t + c^2_t \le w^1_t + w^2_t \]
To solve this, we have the following FOCs ($\forall i, \forall t$):
\[ \frac{\alpha_i \beta^t}{c^i_t} = \lambda_{t}  \]
where $\lambda_t$ is the Lagrange multiplier for the feasibility constraint at time $t$. Using the fact that these constraints must bind (since $\log$ is monotonically increasing), we have
\[ \frac{\alpha_1 \beta^t}{\lambda_{t}} + \frac{\alpha_2 \beta^t}{\lambda_t} = w^1_t + w^2_t \]
\[ \frac{\beta^t(\alpha_1 + \alpha_2)}{w^1_t + w^2_t} = \lambda_t \]
and therefore
\[ c^i_t = \frac{\alpha_i}{\alpha_1 + \alpha_2}(w^1_t + w^2_t) \]
Using the fact that
$ \vector{w^1} + \vector{w^2} = \{4, 4, 4, 4, \dots\} $
we can simplify this to
\[ c^i_t = \frac{4\alpha_i}{\alpha_1 + \alpha_2}\]

\problempart{(3)}
We could compute the demand manually using the optimization problem and FOCs, but we note that this utility is of the Cobb-Douglas form, and hence each consumer spends a proportion of their wealth on the good according to the Cobb-Douglas coefficient. We first define
\[ e_1(\vector{p}) = p_0 + 3p_1 + p_2 + 3p_3 + ...  \]
\[ e_2(\vector{p}) = 3p_0 + p_1 + 3p_2 + p_3 + ...  \]
For convenience, we note
\[ 1 + \beta + \beta^2 + ... = \frac{1}{1-\beta}  \]
Then the demand of consumer of type $i$ is given by
\[ [x^i(\vector{p})]_t = \frac{\beta^t(1-\beta)e_i(\vector{p})}{p_t} \]
Setting excess demand for goods at each period to 0, we get
\[ \frac{\beta^t(1-\beta)e_1(\vector{p})}{p_t} + \frac{\beta^t(1-\beta)e_2(\vector{p})}{p_t} = 4 \]
Using the fact that $e_1(\vector{p}) + e_2(\vector{p}) = 4p_0 + 4p_1 + 4p_2 + ...$, we have
\[ \beta^t(1-\beta)\sum_{t'=0}^\infty p_{t'} = p_t \]
Fixing $\tilde{p_0} = p$, we have that the equilibrium price vector is given by
\[\vector{\tilde{p}} = \{ \beta^t p \}_{t=0}^\infty \]
Then we get
\[ e_1(\vector{\tilde{p}}) = \frac{p}{1-\beta} + \frac{2\beta p}{1-\beta^2} = \frac{p + 3\beta p}{1-\beta^2} \]
\[ e_2(\vector{\tilde{p}}) = \frac{p}{1-\beta} + \frac{2p}{1-\beta^2} = \frac{3p + \beta p}{1-\beta^2} \]
Together, the equilibrium is then:
\[ \left\{\vector{\tilde{c^1}} = \left\{\frac{1+3\beta }{1+\beta} \right\}_{t=0}^\infty, \vector{\tilde{c^2}} = \left\{\frac{3+\beta }{1+\beta} \right\}_{t=0}^\infty, \vector{\tilde{p}} =\{ \beta^t p \}_{t=0}^\infty \right\} \]
To show that this is Pareto efficient, it suffices to show that this is a solution to the social planner problem for some $(\alpha_1, \alpha_2)$ (since the solution to the social planner problem is Pareto efficient). Using the previous problem part, we easily see that choosing $(\alpha_1, \alpha_2) = (1+3\beta, 3+\beta)$ suffices.
\problempart{(4)}
\problempart{(5)}

\problem{2}
\problempart{(1)}
\problempart{(2)}
\problempart{(3)}
\problempart{(4)}
\problempart{(5)}

\problem{3}
\problempart{(1)}
\problempart{(2)}
\end{document}
	% line of code telling latex that your document is ending. If you leave this out, you'll get an error
